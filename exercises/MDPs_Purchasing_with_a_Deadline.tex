\subsection{Purchasing with a deadline}
Consider the following scenario: as the purchasing manager of a factory, you have $T$ days to buy a certain material that is required for the production process. At the beginning of each day $k$, you observe a price offer for the material $x_k \sim \mathit{Uniform}[a,b]$, drawn independently of the prices at other days. You may decide to either buy the material at the price $x_k$ (and then the scenario ends), or wait for the next day. If a purchase hasn't been made by day $T$, the product has to be purchased at the last price $x_T$.

The goal is to decide when to buy the material, such that its price (in expectation) is lowest.

1. Explain intuitively, why the optimal decision whether to buy at some price $x$ may be different at different days.

We formulate the problem as a finite horizon MDP as follows. The state space at time $k$ is $x_k\in [a,b]$. The action space is binary $a_k \in \{buy,hold\}$. Once a $buy$ action is executed at state $x_k$, an immediate cost of $x_k$ is incurred, and the state transitions to a terminal state $x^*$ with zero cost. Otherwise, if a   $hold$ action is executed, no cost is incurred, and the state transitions to $x_{k+1} \sim \mathit{Uniform}[a,b]$. At time $k=T$, the only available action is $buy$. Also, once the terminal state is reached, the state no longer changes, and no cost is further incurred.

2. Define the optimal value function $V_k(x) \doteq \inf_\pi \mathbb{E}^\pi \left[ \sum_{t=k}^{T} c(x_t,\pi(x_t)) | x_k = x\right]$. Note that by definition $V_k(x^*)=0$ for all $k$.

%\quad a. What is $V_k(x^*)$?

\quad a. What is $V_T(x)$?

\quad b. Write a recursive equation for $V_k(x)$ as a function of  $V_{k+1}(x)$.

3. For $a=0,b=1$, draw the value functions $V_T(x)$, $V_{T-1}(x)$, and $V_{T-2}(x)$.

4. Prove that the optimal policy is a threshold policy, given by
\begin{equation*}
\left\{
  \begin{array}{ll}
    buy, & \text{if } x_k \leq \alpha_k \\
    hold, & \text{if } x_k > \alpha_k.
  \end{array}
\right.
\end{equation*}

\quad Write a recursive equation for $\alpha_k$.

\vspace{20pt}
From now on, we consider a scenario where there is correlation between the prices at different days. Specifically:
\begin{equation*}
    x_{k+1} = \lambda x_k + \epsilon_k,
\end{equation*}
where $0 \leq \lambda < 1$ is a known constant, and $\{\epsilon_k\} \in [0,b]$ are i.i.d. random variables with a known distribution $p(\epsilon)$. Let $\bar{\epsilon} = \mathbb{E} [\epsilon_k ]$, and assume that $\bar{\epsilon} > 0$. The state space in this case is $x_k\in \mathbb{R}$.

5. Write a recursive equation for the value function $V_k(x)$ for this scenario.

\vspace{10pt}

We will now show that in the correlated case as well, the optimal policy is a threshold policy. We will do this in several steps.

A function $f(x)$ is said to be \emph{concave} if, for any $x,y$ and $\beta \in [0,1]$, it holds that $f(\beta x + (1-\beta) y) \geq  \beta f(x) + (1-\beta) f(y)$.

6. Prove that if $f(x)$ is concave, then $\mathbb{E} [f(\lambda x + \epsilon_k)]$ is also concave.

7. Prove that for all $1 \leq k \leq T$, the value function $V_k(x)$ is concave.

\quad \textbf{Hint}: you may use the following fact: the minimum of two concave functions is concave.

%A function $f(x)$ is said to be \emph{increasing} if, for any $x<y$ it holds that $f(x) < f(y)$.

8. * Prove that the optimal policy is again a threshold policy.

\quad \textbf{Hint}: You may first prove that for all $1 \leq k \leq T$, $V_k(x)$ is increasing, and positive for $x\geq 0$. 